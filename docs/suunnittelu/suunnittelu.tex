\documentclass[gradu,emptyfirstpagenumber]{tktltiki}
\usepackage{url}
\usepackage{graphicx}

\begin{document}

\title{GREP - Suunnitteludokumentti}
\author{Sami Saada - saada@cs.helsinki.fi}
\date{\today}
\level{Tietorakenteiden harjoitusty�}

\maketitle

\doublespacing

\faculty{Matemaattis-luonnontieteellinen}
\department{Tietojenk�sittelytieteen laitos}
\depositeplace{}
\additionalinformation{}
\numberofpagesinformation{\numberofpages\ sivua}
\classification{}
\keywords{Tietorakenteiden harjoitusty�, GREP}

\begin{abstract}
Tietorakenteiden harjoitusty�n, jonka aiheena on grep, suunnitteludokumentti.
\end{abstract}

\mytableofcontents

\section{Aiheen kuvaus}

Ohjelmalle sy�tet��n muodollinen s��nn�llinen lauseke, joka m��ritt�� riville tilan, jonka mukaan rivi tulostetaan tai sitten ei.

\section{K�ytett�v�t tietorakenteet ja algoritmit}

\subsection{Tietorakenteet}

Pino, suunnattu verkko, lista, yms. muut tilakoneen vaatimat tietorakenteet.

\subsection{Algoritmit}

S��nn�llisten lausekkeiden muotoilu tilakoneeksi, rivin lukeminen tiedostosta ja tilakoneen k�ytt� riville.

\lastpage
\appendices

\end{document}
